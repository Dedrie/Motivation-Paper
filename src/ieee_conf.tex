%% bare_conf.tex
%% vim: set spell:
%% V1.3
%% 2007/01/11
%% by Michael Shell
%% See:
%% http://www.michaelshell.org/
%% for current contact information.
%%
%% This is a skeleton file demonstrating the use of IEEEtran.cls
%% (requires IEEEtran.cls version 1.7 or later) with an IEEE conference paper.
%%
%% Support sites:
%% http://www.michaelshell.org/tex/ieeetran/
%% http://www.ctan.org/tex-archive/macros/latex/contrib/IEEEtran/
%% and
%% http://www.ieee.org/

%%*************************************************************************
%% Legal Notice:
%% This code is offered as-is without any warranty either expressed or
%% implied; without even the implied warranty of MERCHANTABILITY or
%% FITNESS FOR A PARTICULAR PURPOSE! 
%% User assumes all risk.
%% In no event shall IEEE or any contributor to this code be liable for
%% any damages or losses, including, but not limited to, incidental,
%% consequential, or any other damages, resulting from the use or misuse
%% of any information contained here.
%%
%% All comments are the opinions of their respective authors and are not
%% necessarily endorsed by the IEEE.
%%
%% This work is distributed under the LaTeX Project Public License (LPPL)
%% ( http://www.latex-project.org/ ) version 1.3, and may be freely used,
%% distributed and modified. A copy of the LPPL, version 1.3, is included
%% in the base LaTeX documentation of all distributions of LaTeX released
%% 2003/12/01 or later.
%% Retain all contribution notices and credits.
%% ** Modified files should be clearly indicated as such, including  **
%% ** renaming them and changing author support contact information. **
%%
%% File list of work: IEEEtran.cls, IEEEtran_HOWTO.pdf, bare_adv.tex,
%%                    bare_conf.tex, bare_jrnl.tex, bare_jrnl_compsoc.tex
%%*************************************************************************

% *** Authors should verify (and, if needed, correct) their LaTeX system  ***
% *** with the testflow diagnostic prior to trusting their LaTeX platform ***
% *** with production work. IEEE's font choices can trigger bugs that do  ***
% *** not appear when using other class files.                            ***
% The testflow support page is at:
% http://www.michaelshell.org/tex/testflow/



% Note that the a4paper option is mainly intended so that authors in
% countries using A4 can easily print to A4 and see how their papers will
% look in print - the typesetting of the document will not typically be
% affected with changes in paper size (but the bottom and side margins will).
% Use the testflow package mentioned above to verify correct handling of
% both paper sizes by the user's LaTeX system.
%
% Also note that the "draftcls" or "draftclsnofoot", not "draft", option
% should be used if it is desired that the figures are to be displayed in
% draft mode.
%
\documentclass[conference]{IEEEtran}
% Add the compsoc option for Computer Society conferences.
%
% If IEEEtran.cls has not been installed into the LaTeX system files,
% manually specify the path to it like:
% \documentclass[conference]{../sty/IEEEtran}


\usepackage[bookmarks=true, urlcolor=blue, linkcolor=blue, hidelinks=true, citecolor=blue, colorlinks=true, pdftitle={Human Factors in Software Test ing}, pdfauthor={Rahul Gopinath}]{hyperref}


% Some very useful LaTeX packages include:
% (uncomment the ones you want to load)


% *** MISC UTILITY PACKAGES ***
%
%\usepackage{ifpdf}
% Heiko Oberdiek's ifpdf.sty is very useful if you need conditional
% compilation based on whether the output is pdf or dvi.
% usage:
% \ifpdf
%   % pdf code
% \else
%   % dvi code
% \fi
% The latest version of ifpdf.sty can be obtained from:
% http://www.ctan.org/tex-archive/macros/latex/contrib/oberdiek/
% Also, note that IEEEtran.cls V1.7 and later provides a builtin
% \ifCLASSINFOpdf conditional that works the same way.
% When switching from latex to pdflatex and vice-versa, the compiler may
% have to be run twice to clear warning/error messages.





% *** CITATION PACKAGES ***
%
\usepackage{cite}
\usepackage{ctable}
\usepackage{float}
\restylefloat{table}
% cite.sty was written by Donald Arseneau
% V1.6 and later of IEEEtran pre-defines the format of the cite.sty package
% \cite{} output to follow that of IEEE. Loading the cite package will
% result in citation numbers being automatically sorted and properly
% "compressed/ranged". e.g., [1], [9], [2], [7], [5], [6] without using
% cite.sty will become [1], [2], [5]--[7], [9] using cite.sty. cite.sty's
% \cite will automatically add leading space, if needed. Use cite.sty's
% noadjust option (cite.sty V3.8 and later) if you want to turn this off.
% cite.sty is already installed on most LaTeX systems. Be sure and use
% version 4.0 (2003-05-27) and later if using hyperref.sty. cite.sty does
% not currently provide for hyperlinked citations.
% The latest version can be obtained at:
% http://www.ctan.org/tex-archive/macros/latex/contrib/cite/
% The documentation is contained in the cite.sty file itself.






% *** GRAPHICS RELATED PACKAGES ***
%
\ifCLASSINFOpdf
  % \usepackage[pdftex]{graphicx}
  % declare the path(s) where your graphic files are
  % \graphicspath{{../pdf/}{../jpeg/}}
  % and their extensions so you won't have to specify these with
  % every instance of \includegraphics
  % \DeclareGraphicsExtensions{.pdf,.jpeg,.png}
\else
  % or other class option (dvipsone, dvipdf, if not using dvips). graphicx
  % will default to the driver specified in the system graphics.cfg if no
  % driver is specified.
  % \usepackage[dvips]{graphicx}
  % declare the path(s) where your graphic files are
  % \graphicspath{{../eps/}}
  % and their extensions so you won't have to specify these with
  % every instance of \includegraphics
  % \DeclareGraphicsExtensions{.eps}
\fi
% graphicx was written by David Carlisle and Sebastian Rahtz. It is
% required if you want graphics, photos, etc. graphicx.sty is already
% installed on most LaTeX systems. The latest version and documentation can
% be obtained at: 
% http://www.ctan.org/tex-archive/macros/latex/required/graphics/
% Another good source of documentation is "Using Imported Graphics in
% LaTeX2e" by Keith Reckdahl which can be found as epslatex.ps or
% epslatex.pdf at: http://www.ctan.org/tex-archive/info/
%
% latex, and pdflatex in dvi mode, support graphics in encapsulated
% postscript (.eps) format. pdflatex in pdf mode supports graphics
% in .pdf, .jpeg, .png and .mps (metapost) formats. Users should ensure
% that all non-photo figures use a vector format (.eps, .pdf, .mps) and
% not a bitmapped formats (.jpeg, .png). IEEE frowns on bitmapped formats
% which can result in "jaggedy"/blurry rendering of lines and letters as
% well as large increases in file sizes.
%
% You can find documentation about the pdfTeX application at:
% http://www.tug.org/applications/pdftex





% *** MATH PACKAGES ***
%
%\usepackage[cmex10]{amsmath}
% A popular package from the American Mathematical Society that provides
% many useful and powerful commands for dealing with mathematics. If using
% it, be sure to load this package with the cmex10 option to ensure that
% only type 1 fonts will utilized at all point sizes. Without this option,
% it is possible that some math symbols, particularly those within
% footnotes, will be rendered in bitmap form which will result in a
% document that can not be IEEE Xplore compliant!
%
% Also, note that the amsmath package sets \interdisplaylinepenalty to 10000
% thus preventing page breaks from occurring within multiline equations. Use:
%\interdisplaylinepenalty=2500
% after loading amsmath to restore such page breaks as IEEEtran.cls normally
% does. amsmath.sty is already installed on most LaTeX systems. The latest
% version and documentation can be obtained at:
% http://www.ctan.org/tex-archive/macros/latex/required/amslatex/math/





% *** SPECIALIZED LIST PACKAGES ***
%
%\usepackage{algorithmic}
% algorithmic.sty was written by Peter Williams and Rogerio Brito.
% This package provides an algorithmic environment fo describing algorithms.
% You can use the algorithmic environment in-text or within a figure
% environment to provide for a floating algorithm. Do NOT use the algorithm
% floating environment provided by algorithm.sty (by the same authors) or
% algorithm2e.sty (by Christophe Fiorio) as IEEE does not use dedicated
% algorithm float types and packages that provide these will not provide
% correct IEEE style captions. The latest version and documentation of
% algorithmic.sty can be obtained at:
% http://www.ctan.org/tex-archive/macros/latex/contrib/algorithms/
% There is also a support site at:
% http://algorithms.berlios.de/index.html
% Also of interest may be the (relatively newer and more customizable)
% algorithmicx.sty package by Szasz Janos:
% http://www.ctan.org/tex-archive/macros/latex/contrib/algorithmicx/




% *** ALIGNMENT PACKAGES ***
%
%\usepackage{array}
% Frank Mittelbach's and David Carlisle's array.sty patches and improves
% the standard LaTeX2e array and tabular environments to provide better
% appearance and additional user controls. As the default LaTeX2e table
% generation code is lacking to the point of almost being broken with
% respect to the quality of the end results, all users are strongly
% advised to use an enhanced (at the very least that provided by array.sty)
% set of table tools. array.sty is already installed on most systems. The
% latest version and documentation can be obtained at:
% http://www.ctan.org/tex-archive/macros/latex/required/tools/


%\usepackage{mdwmath}
%\usepackage{mdwtab}
% Also highly recommended is Mark Wooding's extremely powerful MDW tools,
% especially mdwmath.sty and mdwtab.sty which are used to format equations
% and tables, respectively. The MDWtools set is already installed on most
% LaTeX systems. The lastest version and documentation is available at:
% http://www.ctan.org/tex-archive/macros/latex/contrib/mdwtools/


% IEEEtran contains the IEEEeqnarray family of commands that can be used to
% generate multiline equations as well as matrices, tables, etc., of high
% quality.


%\usepackage{eqparbox}
% Also of notable interest is Scott Pakin's eqparbox package for creating
% (automatically sized) equal width boxes - aka "natural width parboxes".
% Available at:
% http://www.ctan.org/tex-archive/macros/latex/contrib/eqparbox/





% *** SUBFIGURE PACKAGES ***
%\usepackage[tight,footnotesize]{subfigure}
% subfigure.sty was written by Steven Douglas Cochran. This package makes it
% easy to put subfigures in your figures. e.g., "Figure 1a and 1b". For IEEE
% work, it is a good idea to load it with the tight package option to reduce
% the amount of white space around the subfigures. subfigure.sty is already
% installed on most LaTeX systems. The latest version and documentation can
% be obtained at:
% http://www.ctan.org/tex-archive/obsolete/macros/latex/contrib/subfigure/
% subfigure.sty has been superceeded by subfig.sty.



%\usepackage[caption=false]{caption}
%\usepackage[font=footnotesize]{subfig}
% subfig.sty, also written by Steven Douglas Cochran, is the modern
% replacement for subfigure.sty. However, subfig.sty requires and
% automatically loads Axel Sommerfeldt's caption.sty which will override
% IEEEtran.cls handling of captions and this will result in nonIEEE style
% figure/table captions. To prevent this problem, be sure and preload
% caption.sty with its "caption=false" package option. This is will preserve
% IEEEtran.cls handing of captions. Version 1.3 (2005/06/28) and later 
% (recommended due to many improvements over 1.2) of subfig.sty supports
% the caption=false option directly:
%\usepackage[caption=false,font=footnotesize]{subfig}
%
% The latest version and documentation can be obtained at:
% http://www.ctan.org/tex-archive/macros/latex/contrib/subfig/
% The latest version and documentation of caption.sty can be obtained at:
% http://www.ctan.org/tex-archive/macros/latex/contrib/caption/




% *** FLOAT PACKAGES ***
%
%\usepackage{fixltx2e}
% fixltx2e, the successor to the earlier fix2col.sty, was written by
% Frank Mittelbach and David Carlisle. This package corrects a few problems
% in the LaTeX2e kernel, the most notable of which is that in current
% LaTeX2e releases, the ordering of single and double column floats is not
% guaranteed to be preserved. Thus, an unpatched LaTeX2e can allow a
% single column figure to be placed prior to an earlier double column
% figure. The latest version and documentation can be found at:
% http://www.ctan.org/tex-archive/macros/latex/base/



%\usepackage{stfloats}
% stfloats.sty was written by Sigitas Tolusis. This package gives LaTeX2e
% the ability to do double column floats at the bottom of the page as well
% as the top. (e.g., "\begin{figure*}[!b]" is not normally possible in
% LaTeX2e). It also provides a command:
%\fnbelowfloat
% to enable the placement of footnotes below bottom floats (the standard
% LaTeX2e kernel puts them above bottom floats). This is an invasive package
% which rewrites many portions of the LaTeX2e float routines. It may not work
% with other packages that modify the LaTeX2e float routines. The latest
% version and documentation can be obtained at:
% http://www.ctan.org/tex-archive/macros/latex/contrib/sttools/
% Documentation is contained in the stfloats.sty comments as well as in the
% presfull.pdf file. Do not use the stfloats baselinefloat ability as IEEE
% does not allow \baselineskip to stretch. Authors submitting work to the
% IEEE should note that IEEE rarely uses double column equations and
% that authors should try to avoid such use. Do not be tempted to use the
% cuted.sty or midfloat.sty packages (also by Sigitas Tolusis) as IEEE does
% not format its papers in such ways.





% *** PDF, URL AND HYPERLINK PACKAGES ***
%
%\usepackage{url}
% url.sty was written by Donald Arseneau. It provides better support for
% handling and breaking URLs. url.sty is already installed on most LaTeX
% systems. The latest version can be obtained at:
% http://www.ctan.org/tex-archive/macros/latex/contrib/misc/
% Read the url.sty source comments for usage information. Basically,
% \url{my_url_here}.





% *** Do not adjust lengths that control margins, column widths, etc. ***
% *** Do not use packages that alter fonts (such as pslatex).         ***
% There should be no need to do such things with IEEEtran.cls V1.6 and later.
% (Unless specifically asked to do so by the journal or conference you plan
% to submit to, of course. )


% correct bad hyphenation here
\hyphenation{op-tical net-works semi-conduc-tor}


\begin{document}
%
% paper title
% can use linebreaks \\ within to get better formatting as desired
\title{Intrinsic Motivation in Software Developers}


% author names and affiliations
% use a multiple column layout for up to three different
% affiliations

\author{
\IEEEauthorblockN{Dedrie Beardsley, Rahul Gopinath}
\IEEEauthorblockA{School of EECS, Oregon State University\\
\emph{\{beardsley, gopinath\}@eecs.oregonstate.edu}}
}

% conference papers do not typically use \thanks and this command
% is locked out in conference mode. If really needed, such as for
% the acknowledgment of grants, issue a \IEEEoverridecommandlockouts
% after \documentclass

% for over three affiliations, or if they all won't fit within the width
% of the page, use this alternative format:
% 
%\author{\IEEEauthorblockN{Michael Shell\IEEEauthorrefmark{1},
%Homer Simpson\IEEEauthorrefmark{2},
%James Kirk\IEEEauthorrefmark{3}, 
%Montgomery Scott\IEEEauthorrefmark{3} and
%Eldon Tyrell\IEEEauthorrefmark{4}}
%\IEEEauthorblockA{\IEEEauthorrefmark{1}School of Electrical and Computer Engineering\\
%Georgia Institute of Technology,
%Atlanta, Georgia 30332--0250\\ Email: see http://www.michaelshell.org/contact.html}
%\IEEEauthorblockA{\IEEEauthorrefmark{2}Twentieth Century Fox, Springfield, USA\\
%Email: homer@thesimpsons.com}
%\IEEEauthorblockA{\IEEEauthorrefmark{3}Starfleet Academy, San Francisco, California 96678-2391\\
%Telephone: (800) 555--1212, Fax: (888) 555--1212}
%\IEEEauthorblockA{\IEEEauthorrefmark{4}Tyrell Inc., 123 Replicant Street, Los Angeles, California 90210--4321}}




% use for special paper notices
%\IEEEspecialpapernotice{(Invited Paper)}




% make the title area
\maketitle


\begin{abstract}

Software engineering requires a large amount of problem solving
perseverance and hard work that relies not only on extrinsic motivation
factors and (expand so it ties into last sentence). In our current
economy, engineers are motivated extrinsically by monetary means or
seeking approval which has been proven to decrease complex problem
solving ability and intrinsic motivation. Intrinsic motivation or the
desire to has largely been ignored as. By ignoring intrinsic factors we
set up developers to fall into a certain work style. Explain personas
Developers can be split up into different personas based on blah blahj .
Using a survey tool we measured the intrinsic motivation level of 100
developers of varying personas. Based on this data we highlight areas
for improving current IDE's to foster increased problem solving and
creativity.
\end{abstract}
% IEEEtran.cls defaults to using nonbold math in the Abstract.
% This preserves the distinction between vectors and scalars. However,
% if the conference you are submitting to favors bold math in the abstract,
% then you can use LaTeX's standard command \boldmath at the very start
% of the abstract to achieve this. Many IEEE journals/conferences frown on
% math in the abstract anyway.

\begin{IEEEkeywords}
Motivation, Psychology
\end{IEEEkeywords}


% For peer review papers, you can put extra information on the cover
% page as needed:
% \ifCLASSOPTIONpeerreview
% \begin{center} \bfseries EDICS Category: 3-BBND \end{center}
% \fi
%
% For peerreview papers, this IEEEtran command inserts a page break and
% creates the second title. It will be ignored for other modes.
\IEEEpeerreviewmaketitle
\section{Introduction}

\begin{itemize}
\item
  Explain how software dev is done and challenges devs face (why do we
  need to help them)
\item
  Tie these pieces into software engineering and innovation and business
  case
\item
  Explain developer personas (People with different needs)
\item
  Explain Intrinsic and extrinsic motivation
\item
  Studies showing extrinsically motivating someone decreases problem
  solving ability
\item
  Self determination theory Autonomy capableness and relatedness
\item
  Explain growth mindset.
\item
  Need to have tools that help peeps have more intrinsic motivation
\item
  Research questions spelled out
\item
  What role does intrinsic motivation play for different kinds of
  developers? (break down into more operations)
\end{itemize}

\section{Related Work}

Human beings need motivation to achieve their heights of creativity. It
is the spark that differentiates the mundane from true artistry. It is a
state of cognition that produces results through initiative, energy,
direction, perseverance and equifinality. It is necessary to understand
this facet of our cognition in order to create interfaces that
facilitate it.

Ryan and Deci suggested the Self determination theory of
motivation\cite{ryan2000self} as a model for motivation in human
beings. According to this theory, motivation is produced by three basic
psychological needs. They are: autonomy, competence, and relatedness.
These are discussed in detail next.

\emph{Autonomy:} The sense of autonomy refers to an internal locus of
causality. The motivation of an individual is graded based on the amount
of autonomy perceived by the individual at the task at hand.

\emph{Relatedness:} This is the sense of belonging or need for inclusion
into a specific group of individuals such as family or a peer group.

\emph{Competence:} This is the perception of the individual that they
can excel at the task in hand.

According to Ryan and Deci, the different states of motivation lie on a
continuum based on the perceived locus of causality. The continuum
begins with amotivation, the state of no motivation, with impersonal
locus, followed by extrinsic motivation, and intrinsic motivation. The
extrinsic motivation is graded into external regulation where the locus
of control is external, involving threats and bribes, followed by
introjected regulation which involves internal rewards and punishments.
Next is identified regulation, where personal importance of the task is
understood, and valued internally. This is followed by integrated
regulation where there is a congruence of the task and self. However,
the satisfaction derived out of the task is still not direct. Beyond
this point is intrinsic motivation, where the individual derives direct
enjoyment from the task.

These needs can interact with each other, and their effect depends on
the socio-cultural context of the individual. Due to this reason, it is
hard to specify the effect a particular feature can have on motivation
without an understanding of the socio-cultural context of the
individual. However, it is prohibitive and impractical to identify the
background of each individual. Rather, we strike a compromise with
personas.

 According to \cite{cooper1999the}, a persona is an archetypal user
who represents a particular hypothesised group of users. These are in
some sense a model for the user we synthesize as a byproduct of
investigation into what a typical user of the software is, and what
their goals are. Personas help us to identify a specific related goals
and characteristics that belong together. In our investigation, personas
help us to collate expectations and requirements of specific users who
have specific cognitive characteristics. Aspects that induce motivation
are : need for variety, problem solving, working to benefit others
(relatedness) and challenge (a function of autonomy and competence).

\begin{itemize}
\item
  Formal definitions context of how it's been studied how am I adding
  new knowledge, anything done showing impact of intrinsic motivation in
  the field.
\end{itemize}

\subsection{Important Works}

\begin{itemize}
\item
  A review of literature on motivation in software engineering
  \cite{beecham2008motivation}. They find that software engineers
  are distinguishable from others, and are motivated according to
  internal factors such as characteristics, personality, and external
  factors such as career stage
\item
  Motivation in software engineering: A systematic review
  update\cite{francca2011motivation}. The review looks at five
  questions: What distinguishes S.E? What motivates them? How to
  identify motivated SE? What aspects of SE motivates them? and what
  models exist? They identify a set of motivators and demotivators.
\item
  Towards an Explanatory Theory of Motivation in Software Engineering: A
  Qualitative Case Study of a Small Software Company
  \cite{francca2012towards}: Authors conducted a structured
  interview in a small company. They identify a set of negative and
  positive characteristics.

  \begin{itemize}
  \item
    Negative Task: Low and High learning, low challenge, low efficacy,
    low task identity
  \item
    Positive task: High learning, high challenge, high efficacy, high
    autonomy
  \item
    Negative teamwork: Low competence
  \item
    Positive teamwork: High collaboration, High integration, High
    motivation, High competence
  \end{itemize}
\item
  Effects of gender, intrinsic motivation, and user perceptions in
  end-user applications at work \cite{chintakovid2009effects}

  \begin{itemize}
  \item
    Interesting: Do the simple and enhanced version of the end-user
    debugging software affect users' intrinsic motivation and
    performance differently? If so, how?
  \item
    Interesting: Is there an interaction effect between the two versions
    of the end-user debugging software and gender on users' intrinsic
    motivation and performance? If so, how?
  \item
    Interesting: Do computer self-efficacy, cognitive playfulness,
    perceived ease of use, and perceived usefulness relate to users'
    intrinsic motivation and performance? If so, how do these factors
    interact in affecting intrinsic motivation and performance in
    end-user debugging?
  \end{itemize}
\item
  The Impact of Curiosity and External Regulation on Intrinsic
  Motivation: An Empirical Study in Hong Kong
  Education\cite{hon2012impact}
\item
  The effects of an intrinsically motivating instructional environment
  on software learning and acceptance\cite{davis1994effects}

  \begin{itemize}
  \item
    Old study looking at difference between UI of Mac and Dos
  \end{itemize}
\item
  Models of motivation in software engineering\cite{sharp2009models}
  (review of lit)

  \begin{itemize}
  \item
    Identifies a set of factors that motivates people
  \end{itemize}
\end{itemize}

\subsection{Theories of motivation}

\begin{itemize}
\item
  Theory of Human Motivation (Maslow's Hierarchy of needs)
  \cite{maslow1943theory}
\end{itemize}

This theory is often represented as a pyramid, such that the largest
layer is the basest needs, and lies at the bottom, and on the very top
is the need for self actualization. The complexity of the need increases
as we go up the pyramid. The lowest level is the physiological
requirements for survival such as food, sleep, and procreation. When
these are met, organisms move up the hierarchy to the need for safety.
Safety while important, is not as demanding as the physiological needs
and hence placed at a higher level. Next up the hierarchy is social
needs such as for love and belongingness, family, friendship and
intimacy. Further up, we find typically human needs of self-esteem,
confidence, and respect. At the very top is the need for self
actualisation, which includes morality, creativity, and are concerned
with personal fulfilment.

\begin{itemize}
\item
  Achieving society\cite{mcclelland1967achieving} McClelland's Needs
  theory
\end{itemize}

He uses empirical methods to show that the cross-national variation in
personality structure influences national economic development. He also
showed that motivation to achieve success was subject to systematic
variation within societies and temporal variation within a society. He
argues that this variation in motivation contributes to the difference
in rate of development between societies irrespective of history,
politico-economical structure, resources or ideologies. (summary from
review: rewrite after reading the book)

\begin{itemize}
\item
  Herzberg's Motivation-Hygiene Theory \cite{herzberg2005motivation}
\end{itemize}

This two factor theory of motivation suggests that there exist two kinds
of factors: Motivation factors and Hygiene factors. The motivation
factors involve job content such as growth, work, responsibility,
achievement, advancement and recognition. These, when present lead to
job satisfaction, how ever, their absence does not lead to demotivate.
The hygiene factors include company policies, administration,
supervision, interpersonal relations, status, working conditions,
security and salary. These factors when are negative, tend to dissatisfy
people, however their positive presence does not satisfy people.

\begin{itemize}
\item
  Vroom's Expectancy theory \cite{vroom1959some} According to Vroom,
  an individual makes choices based on estimates of how well the
  expected results of a given behavior is going to lead to desired
  results. That is, the intensity of an individuals effort depends on
  their expectation of desired outcome. An individual can be motivated
  if they believe that,

  \begin{itemize}
  \item
    There is a positive correlation between efforts and outcome
  \item
    Favorable performance results in desirable reward
  \item
    Reward satisfies an important need,
  \item
    Desire to satisfy the need is strong enough to make the effort
    worthwhile.
  \end{itemize}
\item
  Locke's Goal setting theory
\item
  Hackman and Oldham's Job Characteristics Theory
\item
  Malone's theory of intrinsic motivation \cite{malone1987making}

  \begin{itemize}
  \item
    Challenge

    \begin{itemize}
    \item
      Goals, Uncertain outcome, Performance Feedback, Self Esteem
    \end{itemize}
  \item
    Curiosity

    \begin{itemize}
    \item
      Sensory, Cognitive
    \end{itemize}
  \item
    Control

    \begin{itemize}
    \item
      Contingency, Choice, Power
    \end{itemize}
  \item
    Fantasy

    \begin{itemize}
    \item
      Emotional, Cognitive, Endogeneity
    \end{itemize}
  \item
    Cooperation and Competition
  \item
    Recognition
  \end{itemize}
\end{itemize}

\subsection{Others}

\begin{itemize}
\item
  All authors addressing the 3 principles of autonomy, relatedness and
  capableness
\item
  Ryan
\item
  Marshall
\end{itemize}

\subsection{Market Research supporting personas}

Personas.

\section{Methods}

\subsection{Persona Survey}

\subsection{Self Deterministic Tool}

\section{Results}

\begin{itemize}
\item
  Personas mapped to total motivation score
\item
  Personas mapped to autonomy
\item
  Personas mapped to relatedness
\item
  Personas mapped to capableness
\item
  Who uses what IDE
\item
  Type of development mapped to personas
\end{itemize}

\section{Discussion}

\begin{itemize}
\item
  Examples of IDE's who support different personas (based on survey,
  informal mapping)
\item
  Anything random interesting
\item
  Recommendations for improvement in the IDE environment
\end{itemize}

\section{Conclusion}

This paper presents a survey result of motivation in software
developers, in the context of self determination theory of motivation.
Summary of key findings, why it matters


% trigger a \newpage just before the given reference
% number - used to balance the columns on the last page
% adjust value as needed - may need to be readjusted if
% the document is modified later
%\IEEEtriggeratref{8}
% The "triggered" command can be changed if desired:
%\IEEEtriggercmd{\enlargethispage{-5in}}

% references section

% can use a bibliography generated by BibTeX as a .bbl file
% BibTeX documentation can be easily obtained at:
% http://www.ctan.org/tex-archive/biblio/bibtex/contrib/doc/
% The IEEEtran BibTeX style support page is at:
% http://www.michaelshell.org/tex/ieeetran/bibtex/
\bibliographystyle{IEEEtran}
% argument is your BibTeX string definitions and bibliography database(s)
\bibliography{IEEEabrv,paper}
%
% <OR> manually copy in the resultant .bbl file
% set second argument of \begin to the number of references
% (used to reserve space for the reference number labels box)


% that's all folks
\end{document}


